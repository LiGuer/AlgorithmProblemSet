* 2 Add Two Numbers
	\Problem

	\Code{C++}
		/**
		* Definition for singly-linked list.
		* struct ListNode {
		*     int val;
		*     ListNode *next;
		*     ListNode() : val(0), next(nullptr) {}
		*     ListNode(int x) : val(x), next(nullptr) {}
		*     ListNode(int x, ListNode *next) : val(x), next(next) {}
		* };
		*/
		class Solution {
		public:
				ListNode* addTwoNumbers(ListNode* l1, ListNode* l2) {
						int s = 0;
						ListNode* ans = new ListNode;
						ListNode* ansTmp = ans;
						
						while(l1 !=nullptr || l2 !=nullptr){
								ListNode* node = new ListNode;
								ansTmp->next = node;
								ansTmp = node;
								
								int tmp = 0;
								if(l1 !=nullptr && l2 !=nullptr){
										tmp = l1->val + l2->val + s;
										l1 = l1->next;
										l2 = l2->next;
								}
								else if(l1 !=nullptr){
										tmp = l1->val + s;
										l1 = l1->next;
								}
								else{
										tmp = l2->val + s;
										l2 = l2->next;
								} 
								
								s = 0;
								
								if(tmp >= 10){
										s = 1;
										tmp -= 10;
								}
								
								ansTmp->val = tmp;
						}
						
						if(s==1){
								ListNode* node = new ListNode;
								ansTmp->next = node;
								ansTmp = node;
								ansTmp->val = 1;
						}
						
						ans = ans->next;
						return ans;
				}
		};

* 5 Longest Palindromic Substring
	\Problem 
		(input) $\.a \qu; a_i \in \mathbb Z$
		$
			\max_{\.x \in \mathbb Z^{1:\dim(\.a)}} \qu& n
			s.t. \qu& n = \dim(\.x)
				& x_i = a_{s+i}  \qu; i = 1:n  \tag{子序列约束}
				& x_i = x_{n - i}  \qu; i = 1:n  \tag{回文约束}
		$

	\Algorithm
		- 动态规划
			$
				f(s,e) &= \{\mb
					f(s-1,e+1) + 2 \qu& f(s,e) > 0 \ and\  a_{s-1} = a_{e-1}
					0 \qu& other.
					\me\right.
				f(s,s) &= 1  \tag{初始易知值}
				f(s,s+1) &= 2 \qu;a_s = a_{s+1}
			$
			- $f()$: $a_{s:e}$的回文字数, 不是回文序列则为0.

	\Code{C++}
		class Solution {
		public:
			string longestPalindrome(string s) {
				int n = s.length();

				int f[n][n];

				for (int i = 0; i < n; i++) {
					for (int j = i; j < n; j++) {
						f[i][j] = 0;
					}
				}

				for (int i = 0; i < n; i++) {
					f[i][i] = 1;

					if (i != n - 1 && s[i] == s[i + 1]) {
						f[i][i + 1] = 2;
					}
				}

				for (int i = 0; i < n; i++) {
					int k = 1;
					while (i - k >= 0 && i + k < n && s[i - k] == s[i + k]) {
						f[i - k][i + k] = f[i - k + 1][i + k - 1] + 2;
						k++;
					}
				}

				for (int i = 0; i < n - 1; i++) {
					int k = 1;
					while (s[i] == s[i + 1] && i - k >= 0 && i + 1 + k < n && s[i - k] == s[i + 1 + k]) {
						f[i - k][i + 1 + k] = f[i - k + 1][i + 1 + k - 1] + 2;
						k++;
					}
				}

				int max = 0, I = 0;
				for (int i = 0; i < n; i++) {
					for (int j = i; j < n; j++) {
						if (max < f[i][j]) {
							max = f[i][j];
							I = i;
						}
					}
				}

				return s.substr(I, max);

			}
		};

* 39 Combination Sum
	\Problem
		(input) $\.a \in mathbb Z_+^n, b \in mathbb Z_+ \qu; a_i ≤ b$
		$
			\.x^T \.a = b
			\.x \in \mathbb N^n
		$
	
	\Algorithm

	\Code{C++}
		class Solution {
		public:
				int dot(vector<int>& a, vector<int>& b) {
						int n = a.size(), ans = 0;

						for (int i = 0; i < n; i++) {
								ans += a[i] * b[i];
						}

						return ans;
				}

				vector<vector<int>> combinationSum(vector<int>& candidates, int target) {
						vector<vector<int>> ans;
						vector<int> x;

						int n = candidates.size();
						for (int i = 0; i < n; i++) {
								x.push_back(0);
						}

						fun(0, x, candidates, target, ans);

						return ans;
				}

				void fun(int ind, vector<int>& x, vector<int>& a, int target, vector<vector<int>>& ans) {
						int n = a.size();

						if (ind == n - 1) {

								while (1) {
										int t = dot(x, a);

										for (int i = 0; i < x.size(); i++) {
												printf("%d ", x[i]);
										}printf(">%d \n", t);

										if (t > target) {
												x[ind] = 0;
												return;
										}

										else if (t == target) {
												getAns(x, a, ans);

												x[ind] = 0;
												return;
										}

										x[ind]++;
								}
						}

						while (1) {
								fun(ind + 1, x, a, target, ans);

								x[ind]++;

								int t = dot(x, a);
								if (t > target) {
										x[ind] = 0;
										return;
								}
						}

						x[ind] = 0;
						return;
				}

				void getAns(vector<int>& x, vector<int>& a, vector<vector<int>>& ans) {
						vector<int> t;
						int  n = a.size();

						for (int i = 0; i < n; i++) {
								for (int j = 0; j < x[i]; j++) {
										t.push_back(a[i]);
								}
						}
						ans.push_back(t);
				}
		};


* 45 Jump Game II
	\Problem
		(input) $\.a \qu; a_i \in \mathbb N$
		$
			\min_{\.x \in \mathbb N^{1:(\dim{a}-1)}} \qu& \dim \.x
			s.t. \qu& x_i ≤ a(\sum_{k=0}^{i-1} x_k)
				& \.1^T \.x = \dim \.a - 1
		$

	\Algorithm
		- 动态规划
			$
				f(i) &= \min \{f(i-k) + 1 | k \in 1:\min(a_i, i), f(i-k) > 0 \ when\  i-k > 0\} 
				=> f(i+k) &\gets \min (f(i+k), f(i) + 1)  \qu; k \in 1:\min(a_i, \dim \.a - i)
				f(0) &= 0  \tag{初始易知值}
				f(\dim \.a) &= \dim \.x^*  \tag{答案}
			$

			- $f(k)$: 当$\.1^T \.x = i$时的优化问题的解, 无解时为0.

	\Code{C++}
		class Solution {
		public:
				int jump(vector<int>& nums) {
						int n = nums.size();
						int f[n];

						for (int i = 0; i < n; i++) {
								f[i] = 0xFFFFFFF;
						}
						f[0] = 0;

						for (int i = 0; i < n - 1; i++) {
								for (int k = 1; k <= nums[i]; k++) {
										if (i + k >= n)
											break;

										f[i + k] = min(f[i + k], f[i] + 1);
								}
						}

						return f[n - 1];
				}
		};

* 54 
	\Code{C++}
		class Solution {
		public:
				vector<int> spiralOrder(vector<vector<int>>& matrix) {
						vector<int> ans;
						
						int m = matrix.size(), n = matrix[0].size();
						
						int x = 0, y = 0, dimX = 0, dimY = 1;
						
						for(int i=0;i<m*n;i++){
								ans.push_back(matrix[x][y]);
								matrix[x][y] = 0xFFFFFFF;
								
								if(dimY > 0 && (y + dimY == n || matrix[x][y + dimY] == 0xFFFFFFF)){
										dimX = 1;
										dimY = 0;
								}
								
								else if(dimY < 0 && (y + dimY == -1 || matrix[x][y + dimY] == 0xFFFFFFF)){
										dimX = -1;
										dimY = 0;
								}
								
								else if(dimX > 0 && (x + dimX == m || matrix[x + dimX][y] == 0xFFFFFFF)){
										dimX = 0;
										dimY = -1;
								}
								
								else if(dimX < 0 && (x + dimX == -1 || matrix[x + dimX][y] == 0xFFFFFFF)){
										dimX = 0;
										dimY = 1;
								}
								
								x += dimX;
								y += dimY;
						}
						
						return ans;
				}
		};

* 62 Unique Paths
	\Problem
		(input) $x_0, y_0 \in \mathbb N$
		方块图,$(0, 0) \to (x_0, y_0)$,只能走右、下的所有路径数.

	\Algorithm
		- 动态规划
		$
			f(x, y) &= f(x-1, y) + f(x, y-1)
			f(0, y) &= 1  \tag{初始易知值}
			f(x, 0) &= 1
		$

		- $f(x, y)$: 位置$x, y$处的路径数.

	\Code{C++}
		class Solution {
		public:
				int uniquePaths(int m, int n) {
						int f[m][n];
						
						for(int i = 0;i<m;i++){
								f[i][0] = 1;
						}
						
						for(int i = 0;i<n;i++){
								f[0][i] = 1;
						}
						
						for(int i = 1;i<m;i++){
								for(int j=1;j<n;j++){
										f[i][j] = f[i][j-1] + f[i-1][j];
								}
						}
						
						return f[m-1][n-1];
				}
		};

* 145

	\Code{C++}
		class Solution {
		public:
				bool is(int* a) {
						int t = a[0] * a[4] * a[8] + a[1] * a[5] * a[6] + a[2] * a[3] * a[7]
									- a[2] * a[4] * a[6] - a[1] * a[3] * a[8] - a[0] * a[5] * a[7];
						return t == 0 ? 1 : 0;
				}

				int maxPoints(vector<vector<int>>& points) {
						int n = points.size(), ans = 0;
						if(n == 1)
								return 1;
						if(n == 2)
								return 2;

						int a[9];
						a[2] = a[5] = a[8] = 1;

						for (int i = 0; i < n; i++) {
								a[0] = points[i][0];
								a[1] = points[i][1];

								for (int j = i + 1; j < n; j++) {
										a[3] = points[j][0];
										a[4] = points[j][1];

										int num = 2;

										for (int k = 0; k < n; k++) {

												if (k == i || k == j)
														continue;

												a[6] = points[k][0];
												a[7] = points[k][1];

												if (is(a))
														num++;
										}

										ans = max(num, ans);
								}
						}

						return ans;
				}
		};

* 233 Number of Digit One
	\Problem
		(input) $n \in \mathbb N$
			求所有比$n$小(包括$n$)的正整数中,含数字$1$的个数.

	\Property
		- 第$i$出现$1$的现象, 是一个周期$10^{i+1}$宽$10^i$高$1$的矩形波.
			第$i$位$1$的矩形位于整个周期的第$10^i+1 ~ 2 × 10^i$个数上,
			$
				(\mb
					\text{数位 i} & \text{周期} & \text{波峰宽度} & \text{波峰高度}
					0 & 10 & 1 & 1
					1 & 100 & 10 & 1
					2 & 1000 & 100 & 1
					\vdots & \vdots & \vdots & \vdots
					i & 10^{i+1} & 10^i & 1
				\me)
			$

			eg.
			$
				& \mb 0 & 1 & 2 & 3 & 4 & 5 & 6 & 7 & 8 & 9 \me \tag{一位$1$在第$2$个}
				& \mb 0 & 1 & ... & 10 & ... & 19 & ... & ... & ... & 99 \me \tag{二位$1$在第$11-20$个}
				& \mb 0 & 1 & ... & 100 & ... & 199 & ... & ... & ... & 999 \me \tag{三位$1$在第$101-200$个}
			$
	
	\Algorithm
		- 将$0 ~ n$的数分解为每一位上的$1$的矩形波, 并求和波峰即可.
		- 步骤
			- 首先, 求完成的周期数, 周期数$×$波峰宽 便是第一部分数字$1$的数量.
			- 然后, 分析正在进行的周期, 求余剩下的数的长度rest
				- $rest ≥ 2 × 10^i$ 时, 波峰部分已完成, 直接加波峰宽度即可.
				- $10^i+1 ≤ rest < 2 × 10^i$ 时, 波峰部分正在进行, 加剩余数的数目即可
				- $rest < 10^i+1 $ 时, 波峰部分未开始, 不用计算.
	
	\Code{C++}
		class Solution {
		public:
				long long countDigitOne(long long n) {
						long long ans = 0;

						long long nt = n, dim = 0;
						while (nt != 0) {
								dim++;
								nt /= 10;
						}

						for (long long i = 0; i < dim; i++) {
								// 完成的周期
								long long tmp = (n + 1) / (long long)pow(10, i + 1) * pow(10, i);

								// 正在进行的周期
								long long rest = (n + 1) % (long long)pow(10, i + 1),
										tmp2 = rest / pow(10, i);

								if (tmp2 > 1) {					// 完成一个波峰
										tmp += pow(10, i);
								}
								else if (tmp2 == 1) {		// 正在完成一个波峰
										tmp += rest % (long long)pow(10, i);
								}
																				// 未完成一个波峰 (不用计算)
								ans += tmp;
						}

						return ans;
				}
		};

* 322
	\Problem
		(input) $\.a, b \qu; a_i, b \in \mathbb N$
		$
			\min_{x} \qu& ||\.x||_1 = \.1^T \.x
			s.t. \qu& \.a^T \.x = b
				& x_i \in \mathbb N
		$

	\Algorithm
		- 动态规划
			$
				f(k) &= \{\mb
					f(k - a_i) + 1  \qu& f(k - a_i) > 0 \ or\  k - a_i = 0
					0  \qu& other
				\me\right. \qu& k \in 1:\dim(\.a_i)
				f(0) &= 0  \tag{初始易知值}
				f(b) &= \min_{x} ||\.x||_1  \tag{答案}
			$

			- $f(k)$: 当$b \gets k$时的优化问题的解, 无解时为0.
		
	\Code{C++}
		class Solution {
			public:
				int coinChange(vector<int>& coins, int amount) {
					if (amount == 0)
						return 0;
			
					sort(coins.begin(), coins.end());
			
					int table[amount + 1];
					table[0] = 0;
					for (int i = 1; i < amount + 1; i++) {
						table[i] = 0xFFFFFFF;
					}
			
					int size = coins.size();
			
					for (int i = 1; i <= amount; i++) {
						for (int j = 0; j < size; j++) {
							if (coins[j] > i)
								break;
			
							int tmp = table[i - coins[j]] + 1;
			
							table[i] = table[i] < tmp ? table[i] : tmp;
						}
					}
			
					return table[amount] == 0xFFFFFFF ? -1 : table[amount];
				}
			};